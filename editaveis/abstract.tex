\begin{resumo}[Abstract]
\begin{otherlanguage*}{english}

Handicapped people, in a certain degree, needs continuous monitoring in
order to prevent accidents or other issues. Besides that, in some cases, a
presence of a carer is needed to help with the wheelchair, and to track vital
signals.
Tecnologies in this field are not evolving fast enough, does not solve
these scenarios at the same time, and, even more, are costly.
In this work we present UMISS, a electric wheelchair that tracks vital
signals, notifies critical events, and moves without human intervention.
With UMISS we expect to create a low cost solution, that allows the
patient to securely take care of himself.

   % Patients with reduce motor activity, in a certain degree, need constant observation in order to prevent accidents or the emergence of other problems. Besides, some levels of paraplegia or tetraplegia need a career's presence for the chair movement and for the capture of vital signals in order to monitor his activities. With this in mind, the work aims the project of a electric wheelchair with sensor for reading biological signals from the patient and a remote monitoring system in order to facilitate the chair control and to alert careers and relatives of possible risks to the physical integrity of the patient. 

   \vspace{\onelineskip}
   \noindent 
   \textbf{Key-words}: latex. abntex. text editoration.
 \end{otherlanguage*}
\end{resumo}
