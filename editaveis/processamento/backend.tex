\section{\textit{Backend} \textit{Django}}
Esta sessão tratará sobre o servidor do sistema, também conhecido como \textit{backend}
ou \textit{API} \textit{Django}. Na primeira subsessão, é tratada uma breve descrição sobre as funções
e características deste. A segunda apresenta o \textit{design} e a arquitetura
de desenvolvimento. Por fim, trataremos as instâncias do projeto desenvolvido.


O projeto UMISS-BackEnd
possui entidades que deram formato à Business
\textit{Tier} do projeto, onde são definidas as propriedades do sistema, bem como ele atuará.

No projeto, foi desenvolvido o diagrama de classes, que auxilia na atividade de compreender
o domínio do projeto. Este está no Anexo \ref{anx:server_uml}.

É possível visualizar que existem várias entidades de usuários. Isto foi desenvolvido pois
existem dois tipos de usuários diferentes no sistema. Eles são o Paciente e o Monitor. Estes
possuem atributos e comportamentos diferentes, gerando assim a necessidade de implementação
de dois domínios diferentes.  Da mesma maneira, existem três tipos de sinais corporais que são enviados pelo paciente.
Esses sinais possuem características diferentes, bem como atributos. Dessa maneira, existe
uma entidade generalista chamada \textit{BodySignal} com características comuns de todas. E
foram criados as demais especializações com suas características  específicas. Elas são:
Sinais de temperatura corporal, Sinais de batimento cardíaco e Sinais de resistência galvânica.

Por fim, existem ligações dos usuários com os sinais corporais, sendo que os pacientes
são donos destes. Assim como exitem pacientes que possuem monitores ligados,
gerando um relacionamento cíclico entre as entidades de usuário.
